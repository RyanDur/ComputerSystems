\documentclass[12pt]{article}
\usepackage{fancyhdr}
\pagestyle{fancyplain}

\usepackage{enumitem}
\usepackage[load=abbr]{siunitx}

\begin{document}
	\lhead{Ryan Durling}
	\rhead{Computer Systems Coursework, Part1}

	\begin{enumerate}[itemsep=3em]
		\item{Assume that a particular device transfers data at an average of \SI[per-mode=symbol]{32}{\kilo\byte\per\s} on a continuous basis. Determine what fraction of the processor time is consumed by this I/O device using interrupt-driven I/O in each of the following cases.}\\	
		\begin{enumerate}[itemsep=3em]
			\item{First assume that the device interrupts for every byte and that interrupt processing takes \SI{20}{\mics}.  This includes the overhead of setting up the interrupt service procedure (ISP) and returning from the ISP, as well as the \SI{4}{\mics} it takes to transfer one byte from the controller of the device.}
		
			\item{ Next assume that the controller of the device has two 16-byte buffers and it interrupts the processor whenever one of the buffers is full.}
		
			\item{Assume, in addition to the buffers, that the processor is equipped with a block transfer I/O instruction which speeds up the transfer of a byte to \SI{2}{\mics}.}
		\end{enumerate}
	
		\item{A \SI{2}{\giga \hertz} processor provides an instruction for loading a string of bytes from memory to an internal cache. The fetching and decoding of the instruction takes 10 clock cycles. Thereafter, it takes 5 clock cycles to transfer each byte.}\\
		\begin{enumerate}[itemsep=3em]
			\item{Determine the length (in seconds) of the instruction cycle for the case of a string of 64 bytes.}
		
			\item{What is the worst-case delay for acknowledging an interrupt if the instruction is non-interruptable?}
		
			\item{Repeat the previous item assuming that the instruction can be interrupted at the beginning of each byte transfer.}
		\end{enumerate}
	\end{enumerate}
\end{document}