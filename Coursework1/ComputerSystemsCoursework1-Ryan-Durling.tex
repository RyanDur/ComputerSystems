\documentclass[12pt]{article}
\usepackage{enumitem}
\usepackage{nicefrac}
\usepackage[load=abbr, per-mode=symbol, alsoload=binary]{siunitx}
\usepackage{fancyhdr}
\pagestyle{fancyplain}

\begin{document}
	\lhead{Ryan Durling}
	\rhead{Computer Systems Coursework, Part1}

	\begin{enumerate}[itemsep=3em]
		\item{Assume that a particular device transfers data at an average of \SI{32}{\kilo\byte\per\s} on a continuous basis. Determine what fraction of the processor time is consumed by this I/O device using interrupt-driven I/O in each of the following cases.}\\	
		\begin{enumerate}[itemsep=3em]
			\item{First assume that the device interrupts for every byte and that interrupt processing takes \SI{20}{\mics}.  This includes the overhead of setting up the interrupt service procedure (ISP) and returning from the ISP, as well as the \SI{4}{\mics} it takes to transfer one byte from the controller of the device.}\\ \\
			\fbox{\parbox{\linewidth}{
				\SI{32000}{\byte\per\s} = \SI{32}{\kilo\byte\per\s} = Average data transfer\\
				\SI{2e-5}{\s} = \SI{20}{\mics} = Interrupt time \\ \\
				 $32000 \times (\SI{2e-5})$ = 0.64 \\
				 0.64 = \SI{64}{\percent} = \nicefrac{64}{100} = \nicefrac{16}{25}
			}}
			
			\newpage
			\item{ Next assume that the controller of the device has two 16-byte buffers and it interrupts the processor whenever one of the buffers is full.}\\ \\
			\fbox{\parbox{\linewidth}{				
				\SI{32000}{\byte\per\s} = \SI{32}{\kilo\byte\per\s} = Average transfer\\
				\SI{4e-6}{\s} = \SI{4}{\mics} = transfer time for 1 byte \\
				\SI{1.6e-5}{\s} = \SI{16}{\mics} = ISP time \\
				\SI{16}{\byte} = Buffer size \\ 
				2 = \# of buffers \\ 
				
				(Average buffer transfer) = (Average transfer) $\div$ (\# of buffers) \\
				16000 = $32000 \div 2$ \\	
				
				(\# of interrupts) = (Average buffer transfer) $\div$ (Buffer size) \\
				1000 = 16000 $\div$ 16 \\		
				
				(time to load a buffer) = (transfer time for 1 byte) $\times$ (Buffer size) \\
				64  = 4 $\times$ 16 \\ 
				
				(interrupt time) = (time to load a buffer) + (ISP time) \\
				\SI{8e-5} = 80 = 64 + 16 \\
					
				(percentage for interrupt) = (\# of interrupts) $\times$ (interrupt time) \\			
				0.08 = 1000 $\times$ (\SI{8e-5}) \\
				
				0.08 = \SI{8}{\percent} = \nicefrac{8}{100} = \nicefrac{2}{25}								
			}}
		
			\newpage
			\item{Assume, in addition to the buffers, that the processor is equipped with a block transfer I/O instruction which speeds up the transfer of a byte to \SI{2}{\mics}.}\\ \\
			\fbox{\parbox{\linewidth}{
				\SI{32000}{\byte\per\s} = \SI{32}{\kilo\byte\per\s} = Average transfer\\
				\SI{2e-6}{\s} = \SI{2}{\mics} = transfer time for 1 byte \\
				\SI{1.6e-5}{\s} = \SI{16}{\mics} = ISP time \\
				\SI{16}{\byte} = Buffer size \\ 
				2 = \# of buffers \\ 
				
				(Average buffer transfer) = (Average transfer) $\div$ (\# of buffers) \\
				16000 = $32000 \div 2$ \\	
				
				(\# of interrupts) = (Average buffer transfer) $\div$ (Buffer size) \\
				1000 = 16000 $\div$ 16 \\		
				
				(time to load a buffer) = (transfer time for 1 byte) $\times$ (Buffer size) \\
				32  = 2 $\times$ 16 \\ 
				
				(interrupt time) = (time to load a buffer) + (ISP time) \\
				\SI{4.8e-5} = 48 = 32 + 16 \\
					
				(percentage for interrupt) = (\# of interrupts) $\times$ (interrupt time) \\			
				0.048 = 1000 $\times$ (\SI{4.8e-5}) \\
				
				0.048 = \SI{4.8}{\percent} = \nicefrac{4.8}{100} = \nicefrac{6}{125}					}}

		\end{enumerate}
	
		\item{A \SI{2}{\giga \hertz} processor provides an instruction for loading a string of bytes from memory to an internal cache. The fetching and decoding of the instruction takes 10 clock cycles. Thereafter, it takes 5 clock cycles to transfer each byte.}\\
		\begin{enumerate}[itemsep=3em]
			\item{Determine the length (in seconds) of the instruction cycle for the case of a string of 64 bytes.}\\ \\
			\fbox{\parbox{\linewidth}{
				 $cycles/\s$  = \hertz \\
				\SI{2e9}{\hertz} = \SI{2000000000}{\hertz} = \SI{2}{\giga \hertz} \\
				\SI{10}{\hertz} = fetching and decoding instruction \\ 
				\SI{5}{\hertz} = byte transfer \\ 
				\SI{64}{\byte} = string \\
				
				10 + ($64 \times 5$) = 330 \\
				 \SI{330e-9}{\s}				
			}}
		
			\item{What is the worst-case delay for acknowledging an interrupt if the instruction is non-interruptable?}\\ \\
			\fbox{\parbox{\linewidth}{
			
			}}
		
			\item{Repeat the previous item assuming that the instruction can be interrupted at the beginning of each byte transfer.}\\ \\
			\fbox{\parbox{\linewidth}{
			
			}}

		\end{enumerate}
	\end{enumerate}
\end{document}