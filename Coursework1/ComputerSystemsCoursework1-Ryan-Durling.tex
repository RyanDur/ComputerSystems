\documentclass[12pt]{article}
\usepackage{enumitem}
\usepackage{nicefrac}
\usepackage[load=abbr, per-mode=symbol, alsoload=binary]{siunitx}
\usepackage{fancyhdr}
\pagestyle{fancyplain}

\begin{document}
	\lhead{Ryan Durling}
	\rhead{Computer Systems Coursework, Part1}

	\begin{enumerate}[itemsep=3em]
		\item{Assume that a particular device transfers data at an average of \SI{32}{\kilo\byte\per\s} on a continuous basis. Determine what fraction of the processor time is consumed by this I/O device using interrupt-driven I/O in each of the following cases.}\\	
		\begin{enumerate}[itemsep=3em]
			\item{First assume that the device interrupts for every byte and that interrupt processing takes \SI{20}{\mics}.  This includes the overhead of setting up the interrupt service procedure (ISP) and returning from the ISP, as well as the \SI{4}{\mics} it takes to transfer one byte from the controller of the device.}\\ \\
			\fbox{\parbox{\linewidth}{
				\SI{32000}{\byte\per\s} = \SI{32}{\kilo\byte\per\s} = Average data transfer\\
				\SI{2e-5}{\s} = \SI{20}{\mics} = Interrupt time \\ \\
				 $32000 \times (\SI{2e-5})$ = 0.64 \\
				 0.64 = \SI{64}{\percent} = \nicefrac{64}{100} = \nicefrac{16}{25}
			}}
			
			\item{ Next assume that the controller of the device has two 16-byte buffers and it interrupts the processor whenever one of the buffers is full.}\\ \\
			\fbox{\parbox{\linewidth}{
				Assumption: It can transfer all bytes from the buffer in the same amount of time as transferring one at a time from the previous question.\\ 
				
				\SI{32000}{\byte\per\s} = \SI{32}{\kilo\byte\per\s} = Average data transfer\\
				\SI{2e-5}{\s} = \SI{20}{\mics} = Interrupt time\\
				\SI{16}{\byte} = Buffer size \\ 
				2 = \# of buffers \\ 
				
				$32000 \div 16$ = 2000 \\
				$2000 \times 2$ = 4000 \\
				$4000 \times (\SI{2e-5})$ =  0.08 \\ \\
				0.08 = \SI{8}{\percent} = \nicefrac{8}{100} = \nicefrac{4}{50} = \nicefrac{2}{25}
			}}
		
			\item{Assume, in addition to the buffers, that the processor is equipped with a block transfer I/O instruction which speeds up the transfer of a byte to \SI{2}{\mics}.}\\ \\
			\fbox{\parbox{\linewidth}{
				\SI{2}{\mics} = Transfer time\\
				\SI{16}{\mics} = ISP setup time \\ 
				\SI{32000}{\byte\per\s} = \SI{32}{\kilo\byte\per\s} = Average data transfer\\
				\SI{1.8e-5}{\s} = \SI{18}{\mics} = \SI{16}{\mics} + \SI{2}{\mics} = Interrupt time\\
				\SI{16}{\byte} = Buffer size \\ 
				2 = \# of buffers \\ 

				$32000 \div 16$ = 2000 \\
				$2000 \times 2$ = 4000 \\
				$4000 \times (\SI{1.8e-5})$ =  0.072 \\ \\
				0.072 = \SI{7.2}{\percent} = \nicefrac{7.2}{100} = \nicefrac{9}{125}
			}}

		\end{enumerate}
	
		\item{A \SI{2}{\giga \hertz} processor provides an instruction for loading a string of bytes from memory to an internal cache. The fetching and decoding of the instruction takes 10 clock cycles. Thereafter, it takes 5 clock cycles to transfer each byte.}\\
		\begin{enumerate}[itemsep=3em]
			\item{Determine the length (in seconds) of the instruction cycle for the case of a string of 64 bytes.}\\ \\
			\fbox{\parbox{\linewidth}{
			
			}}
		
			\item{What is the worst-case delay for acknowledging an interrupt if the instruction is non-interruptable?}\\ \\
			\fbox{\parbox{\linewidth}{
			
			}}
		
			\item{Repeat the previous item assuming that the instruction can be interrupted at the beginning of each byte transfer.}\\ \\
			\fbox{\parbox{\linewidth}{
			
			}}

		\end{enumerate}
	\end{enumerate}
\end{document}